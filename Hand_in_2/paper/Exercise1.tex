\paragraph{Problem:} A spherical shell of radius $R_a$ carries a surface charge $\sigma = \alpha \cos\theta$.
\begin{enumerate}[label=(\alph*)]
    \item Calculate its dipole moment with respect to the center of the sphere.
    \item Calculate the approximate potential at points far away from the sphere.
    \item The exact solution is given by
            \begin{equation}
                V(r,\theta) = \frac{\alpha R_a^3}{3\epsilon_0 r^2} \cos\theta
            \end{equation}
            What does this imply for the higher poles?
\end{enumerate}

\paragraph{Solution:} We start by drawing a sketch seen in Fig. \ref{fig:sketch}.

\begin{figure}[H]
    \centering\hspace*{2.3cm}
    \input{figures/sketch.pdf_tex}
    \caption{Sketch of the problem setup.}
    \label{fig:sketch}
\end{figure}

Since we are looking at a spherical shell it is reasonable to use spherical coordinates, as can be seen in the sketch. The cartesian axes are shown as well. 

\paragraph{(a) Solution:} The dipole moment $\mathbf{p}$ of a continuous distribution centered at the origin is 
\begin{equation}
    \mathbf{p} = \int_V \mathbf{r}' \rho(\mathbf{r}')\dd \tau',\label{eq:p_start}
\end{equation}
where $\rho$ is the volume charge density, and $V$ is the volume we are integrating over. In our case, since the charge is constricted on the surface, we can use a delta function.
\begin{equation}
    \rho(\mathbf{r}) = \delta(r-R_a)\sigma = \delta(r-R_a)\alpha\cos\theta.
\end{equation}
Using the fact that the vector $\mathbf{r}$ and the volume element $\dd \tau$ in spherical coordinates is $\mathbf{r} = r\mathbf{\hat{r}}$ and $\dd \tau = (\dd r)(r \dd \theta)(r \sin\theta \dd\varphi)$ we can rewrite Eq. \eqref{eq:p_start}.