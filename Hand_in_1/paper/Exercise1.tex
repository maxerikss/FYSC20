\paragraph{Problem:} Find the electrostatic field a distance $s$ from the axis of an infinitely long straight cylinder with radius $a$ that carries a uniform charge density $\rho_0$. Consider both $s < a$ (inside the cylinder) and $s > a$ (outside the cylinder).

\paragraph{Solution:} We can start by drawing a sketch of the problem and define the coordinates. Since the cylinder is infinitely long there are no edge effects and we have cylindrical symmetry. Thus it is sensible to use cylindrical coordinates. We can see the sketch in Fig. \ref{fig:sketch}. Then we can use Gauss's law to calculate the electrostatic field. We draw a Gaussian surface with length $L$, radius $s$ and center at $z=0$. The Gaussian surface can be seen in Fig. \ref{fig:gauss}.

\begin{figure}[H]
    \centering
    \input{figures/problem.pdf_tex}
    \caption{A drawing of the problem with both Cartesian and Cylindrical coordinate axis. The blue axis are Cartesian coordinate while the red axis are the Cylindrical coordinates. The $z$-axis is common.}\label{fig:sketch}
\end{figure}

\begin{figure}[H]
    \centering
    \input{figures/Gauss.pdf_tex}
    \caption{Cylindrical Gaussian surface drawn at radius $s$ with length $L$.}\label{fig:gauss}
\end{figure}

Gauss's law tells us that 
\begin{equation}
    \nabla \cdot \textbf{E}(\vec{r}) = \frac{\rho(s)}{\varepsilon_0},\label{eq:gauss}
\end{equation}
where $\textbf{E}(\vec{r})$ is the electrostatic field, $\varepsilon_0$ is the vacuum permittivity, and $\rho(s)$ is the charge density defined as
\begin{singlespace}
\begin{equation}
    \rho(s) = 
    \begin{cases}
        \rho_0 & s \leq a\\
        0 & s > a.
    \end{cases}
\end{equation}
\end{singlespace}
Rewriting Eq. \eqref{eq:gauss} into integral form over the volume $V$ confined by the Gaussian surface $\partial V$ we get
\begin{equation}
    \int_{V} (\nabla \cdot \vec{E}(\vec{r})) \dd \tau = \frac{1}{\varepsilon_0} \int_V  \rho(s)\dd \tau.
\end{equation}
Applying Gauss's theorem on the left hand side we obtain
\begin{equation}
    \int_{\partial V}  \vec{E}(\vec{r}) \cdot \dd \vec{a} = \frac{1}{\varepsilon_0} \int_V  \rho(s)\dd \tau.\label{eq:intGauss}
\end{equation}
Due to cylindrical symmetry we can write 
\begin{equation}
    \vec{E}(\vec{r}) = E(s) \vec{\hat{s}}.
\end{equation}


Rewriting the left hand sides of Eq. \eqref{eq:intGauss} gives us
\begin{equation}
    E(s) \int_{\partial V} \vec{\hat{s}} \cdot \dd \vec{a} = E(s)s \int_{\partial V} \dd \varphi \dd z =  \frac{1}{\varepsilon_0} \int_V  \rho(s)\dd \tau.
\end{equation}
Due to the symmetry we have the flux through the Gaussian surface at the top and bottom is zero, and only the lateral surface contributes.
\begin{equation}
    E(s)s \int_{\partial V} \dd \varphi \dd z = E(s)s \int_{0}^{2\pi}\dd\varphi \int_{-L/2}^{L/2} \dd z = E(s)s \cdot 2\pi L.
    \label{eq:LHS}
\end{equation}
For the right hand side we must take the integral for the cases $s \leq a$ and $s > a$.
\paragraph{$s \leq a$:} 
\begin{equation}
    \frac{1}{\varepsilon_0} \int_{V} \rho(s) \dd \tau = \frac{\rho_0}{\varepsilon_0} \int_{0}^{2\pi}\dd \varphi \int_{-L/2}^{L/2} \dd z \int_{0}^{s} s' \dd s' = \frac{\rho_0}{\varepsilon_0} \cdot 2\pi L \left[\frac{s'^2}{2}\right]_0^s = \frac{\rho_0}{\varepsilon_0}\cdot \pi L s^2\label{eq:RHS1}
\end{equation}
\paragraph{$s > a$:}
\begin{align}
    \frac{1}{\varepsilon_0} \int_{V} \rho(s) \dd \tau &= \frac{1}{\varepsilon_0} \int_{0}^{2\pi}\dd \varphi \int_{-L/2}^{L/2} \dd z \int_{0}^{s} \rho(s') s' \dd s'\\
    &= \frac{1}{\varepsilon_0} \int_{0}^{2\pi}\dd \varphi \int_{-L/2}^{L/2} \dd z \left( \int_{0}^{a} \rho_0 s' \dd s' + \int_{a}^{s} 0 \cdot s' \dd s' \right)\\
    &= \frac{\rho_0}{\varepsilon_0} \cdot 2 \pi L \left[\frac{s'^2}{2}\right]_0^a = \frac{\rho_0}{\varepsilon_0} \cdot \pi L a^2\label{eq:RHS2}
\end{align}
For the field inside the cylinder we combine Eq. \eqref{eq:LHS} and \eqref{eq:RHS1}
\begin{equation}
    E_\text{in}(s)s \cdot 2\pi L = \frac{\rho_0}{\varepsilon_0}\cdot \pi L s^2 \iff E(s)_\text{in} = \frac{\rho_0 s}{2 \varepsilon_0}  \iff \vec{E}_\text{in}(\vec{r}) = \frac{\rho_0 s}{2 \varepsilon_0} \vec{\hat{s}}.
\end{equation}
For the field outside the cylinder we combine \eqref{eq:LHS} and \eqref{eq:RHS2}
\begin{equation}
    E_\text{out}(s)s \cdot 2\pi L = \frac{\rho_0}{\varepsilon_0} \cdot \pi L a^2 \iff E_\text{out}(s) = \frac{\rho_0 a^2}{2 \varepsilon_0 s} \iff \vec{E}_\text{out} (\vec{r}) = \frac{\rho_0 a^2}{2 \varepsilon_0 s} \vec{\hat{s}}
\end{equation}

\newpage
\paragraph{Answer:} The electrostatic field is
\begin{singlespace}
\begin{equation}
    \vec{E}(\vec{r}) = 
    \begin{cases}
        \vec{E}_\text{in}(\vec{r}) & s \leq a\\
        \vec{E}_\text{out} (\vec{r}) & s > a
    \end{cases}
    = \frac{\rho_0}{2 \varepsilon_0} \vec{\hat{s}} 
    \begin{cases}
        s & s \leq a\\
        a^2/s & s > a
    \end{cases}
\end{equation}
\end{singlespace} 